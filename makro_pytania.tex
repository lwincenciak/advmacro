%& --translate-file=cp1250pl
\documentclass[a4paper, notitlepage, 11pt]{article}
\usepackage[left=2.5cm, right=2.5cm, top=2.5cm, bottom=2.5cm, footskip=1.25cm]{geometry}
\usepackage[OT4]{fontenc}
\usepackage{color}
\usepackage{leon}
\usepackage{fancyhdr}
\usepackage{eurosym}
\usepackage{amsmath}
\usepackage{amssymb}
\usepackage{graphpap}
\usepackage{graphicx}
\usepackage{multirow}
\usepackage{epstopdf}

%-------------------------------------------------------------------------------

\newcommand{\url}[1]{\texttt{#1}}
\newcounter{zadlicz}[section]%

\newcommand{\tytul}[2]{\setcounter{equation}{0}\addtocounter{zadlicz}{1}\vspace{\abovedisplayskip}\noindent\textbf{#1\ \thezadlicz #2}}%

\renewcommand{\theequation}{\arabic{zadlicz}.\arabic{equation}}%

%-------------------------------------------------------------------------------

\title{Advanced Macroeconomics Problem Set}

\author{Leszek Wincenciak, Ph. D.\\ University of Warsaw\\ ver. 17.3.9.003}

\date{\today}

\begin{document}

\maketitle

%-------------------------------------------------------------------------------
\tytul{Problem}{}

\noindent%
The utility of a~consumer is given by $U(C,L)=\alpha\ln C + (1-\alpha)\ln L$, where $C$ is the aggregate consumption, and $L$ is the leisure. Total time available in one week is denoted by $T$, market wage by $w$. Assume that the non-wage income is given by $R$.

\begin{wylicz}
\item Write the budget constraint for this consumer.
\item Set up the utility maximization problem and derive the function of labour supply. Find the expression for the reservation wage.
\item How does the labour supply respond to changes in the non-wage income?
\item How does the labour supply respond to changes in the wage? Is the labour supply curve ,,backward-bending''?
\item If the non-wage income is $R=0$, what would be the answer to point (d)?
\end{wylicz}

Assume that the non-wage income is $R=100$ \$/week, while $\alpha=0.125$. If the market wage is 10 \$/hour, what is the optimal labour supply (hours of work per week) of this consumer? If the working week is set to 40 hours and regulations make it impossible to work part-time, should the consumer take up such a~job or remain inactive?

%-------------------------------------------------------------------------------
\tytul{Problem}{}

\noindent%
Suppose the Phillips curve is given by the following relation:

\begin{equation}
\Delta w_t=\lambda_0 + (1-\lambda_1)\Delta p_t + \lambda_1\Delta p_{t-1}-\lambda_2 u_t + \lambda_3\Delta a_t,\label{phillips}
\end{equation}

and the price setting relation is:

\begin{equation}
\Delta p_t=\Delta w_t - \Delta a_t,\label{price}
\end{equation}

where $w_t$ -- log of wages, $p_t$ -- log of prices, $u_t$ -- unemployment rate, $a_t$ -- log of productivity, $\Delta x_t=x_t-x_{t-1}$ -- change of variable $x$.

Suppose you have the following estimates of the parameters: $\lambda_0=0.03$, $\lambda_1=0.5$, $\lambda_2=0.4$, $\lambda_3=0.2$. Assume that the productivity grows at a~constant rate of 2\% per year.

\begin{wylicz}
\item Using equations \eqref{phillips} and \eqref{price} write the Phillips curve in the form of relation between change in inflation rate and the level of unemployment. Find the expression for NAIRU. Use your estimates and find the value of NAIRU.
\item How would NAIRU change if productivity growth rate slowed down to 1\%? What would happen to inflation rate and unemployment in the short run during the transition period?
\item Using expression from point (b) write the level of current unemployment rate as a~function of NAIRU and the change in inflation. Using your estimates calculate so called 'sacrifice ratio'.
\end{wylicz}

%-------------------------------------------------------------------------------
\tytul{Problem}{}

\noindent%
Assume that the parameters of the Phillips curve have been estimated to: $\lambda_0=0.02$, $\lambda_1=0.5$, $\lambda_2=0.2$, $\lambda_3=0.4$. Assume that the productivity grows at the constant yearly rate $\Delta a=0.02$, while the money supply at $\Delta m=0.05$.

Write in a~homogenous form the system of equations for the dynamics of inflation and unemployment in the NAIRU model. Using the spreadsheet application write appropriate formulas to calculate the values of unemployment and inflation in subsequent moments of time. Set $u_0=0.04$ and $\Delta p_0=0$ as initial values.

\begin{wylicz}
\item Using your spreadsheet make a~scatterplot illustrating the long-run equilibrium and the process of convergence towards it.
\item Set the initial values of $u_0$ and $\Delta p_0$ equal to their long-run equilibrium values. Analyse what will happen to the long-run equilibrium values of unemployment and inflation and how the process of convergence will look like when:
    \begin{wylicz}
      \item[(i)] the growth rate of money supply is reduced to 0.04 per year;
      \item[(ii)] the growth rate of productivity is reduced to 0.01 per year.
    \end{wylicz}
\item[(c)] Reset the values of all parameters. Analyse what happens to the speed of convergence when you change the value of $\lambda_1$.
\end{wylicz}

%-------------------------------------------------------------------------------
\tytul{Problem}{}

\noindent%
Show formally stability of the following system of difference equations:

\[
\left\{%
\begin{array}{l}
    u_t = \overline{u}_t-\frac{\lambda_1}{\lambda_2}(\Delta p_t-\Delta p_{t-1}) \\
    \Delta p_t = \pi + u_t - u_{t-1} \\
\end{array}%
\right.
\]


%-------------------------------------------------------------------------------
\tytul{Problem}{}

\noindent%
Consider the basic efficiency wages model. Suppose that fraction $f$ of workers belong to unions that are able to obtain a~wage that exceeds the nonunion wage by $\mu$ percent, so that $w_u=(1+\mu)w_n$, where subscripts $u$ and $n$ denote union and nonunion wages, respectively. The average wage in the economy can be calculated as $w_a=fw_u+(1-f)w_n$. Suppose the effort function is defined as:

\[
e(w,x)=\label{example_9.12}
\begin{cases}
 \left(\frac{w-x}{x}\right)^\beta & \text{if } w>x \\
 0                                & \text{otherwise},
\end{cases}
\]

where  $x = (1-bu)w_a$ is a~measure of labour market conditions, $0<\beta<1$ and $b>0$.

\begin{wylicz}
\item Find the equilibrium unemployment rate in terms of the exogenous parameters of the model ($\beta,b,f,\mu$).
\item Suppose that $\mu=f=0.15$. What is the equilibrium unemployment rate if $\beta=0.06$ and $b=1$? Is the effort exerted by unionized workers higher than nonunionized workers? By how much? How does it compare to the their wages ratio?
\item What is the cost of effective labor for unionized workers relative to nonunionized workers?
\end{wylicz}

%-------------------------------------------------------------------------------
\tytul{Problem}{}

\noindent%
Consider a~simple model of shirking. Assume that providing effort $e$ causes for a~worker a~cost $\phi(e)$. If the worker is providing enough effort he is paid the wage $w$. In case of shirking, the worker is either caught and fired, or not caught and kept in the workplace despite of the too low productivity. The probability that one is caught for shirking in the firm is a~constant $\theta\in (0, 1)$, and the elasticity of the worker�s efficiency cost is a~constant $1/\varepsilon$:

\begin{equation}
\phi(e)=\mu e^{1/\varepsilon},\qquad \frac{d\phi}{de}\frac{e}{\phi}=\frac{1}{\varepsilon},\quad \mu>0,\quad \varepsilon\in (0,1).\label{phi}\notag
\end{equation}

\begin{wylicz}
\item Write the worker�s income without shirking.
\item Write the expected worker�s income with shirking if the worker who is caught for shirking is fired and the expected wage outside the firm is $v$.
\item State the non-shirking condition and find the wage such that it makes the worker choose not to shirk.
\item Using the non-shirking condition and the cost of the effort function, find the function of the effort dependent on wage.
\item Using Solow condition find the optimum wage. What is the unemployment rate if $v=(1-u)w+ub$, and benefits $b$ are paid as a~constant proportion of the average wage, so that $b=\beta w$?
\end{wylicz}

%-------------------------------------------------------------------------------
\tytul{Problem}{}

\noindent%
Describe how each of the following events affect the equilibrium employment and wage in the Shapiro-Stiglitz model. Use the graph and give an intuitive explanation.

\begin{wylicz}
 \item An increase in workers' discount rate
 \item An increase in the job breakup rate
 \item An improvement in the shirking detection technology
 \item An increase in the size of the labour force
\end{wylicz}

%-------------------------------------------------------------------------------
\tytul{Problem}{}

\noindent%
Assume that in Shapiro-Stiglitz model unemployed workers are employed not at random but according to the length of unemployment period, specifically -- those with longest unemployment spells are employed as first.

\begin{wylicz}
 \item Analyse the steady-state without shirking. Find the expression for average time until finding new job by the unemployed as a~function of $b$, $L$, $N$ and  $\overline L$.
 \item Let $V_U$ be the value of being just unemployed. Find the formula for $V_U$ as a~function of time needed to find another job, workers' discount rate ($\rho$) and the value of being employed ($V_E$).
 \item Using answers from (a) and (b) find the non-shirking condition.
 \item How does the unemployment rate in this version of the model compare to the standard Shapiro-Stiglitz model?
\end{wylicz}

%-------------------------------------------------------------------------------
\tytul{Problem}{}

\noindent%
Analyse Shapiro-Stiglitz model of efficiency wages. In addition to version of the model analyzed in the lecture, let's introduce unemployment benefits $z$, which increase $V_U$ -- the utility of the Unemployed.

\begin{wylicz}
\item Reformulate the asset equation for the value of $V_U$.
\item Find the formula for the non-shirking condition. How does the parameter $z$ influence incentives to provide effort and unemployment in equilibrium?
\item Is it possible that the model generates full employment equilibrium? Use your formula from point (a) to give one example of such situation and show it graphically.
\end{wylicz}

%-------------------------------------------------------------------------------
\newpage%
\tytul{Problem}{}

\noindent%
Assume that in Shapiro-Stiglitz model of efficiency wages, dismissed workers are paid severance payments $F$. For the workers who are caught shirking however, the severance payment is paid with probability $p$ only.

\begin{wylicz}
\item Write the Bellman equations for the values of $V_E$, $V_S$ and $V_U$ and find the 'No Shirking Condition'. For convenience of notation use $r$ as the discount rate.
\item How does increase of $F$ influence unemployment if $p=0$? How does $p$ affect the response of unemployment to $F$? Explain the economic mechanism behind it.
\item Is unemployment in this modified version of the model higher than in the original Shapiro-Stiglitz model? Explain.
\end{wylicz}

%-------------------------------------------------------------------------------
\tytul{Problem}{}

\noindent%
Assume that in Shapiro-Stiglitz model of efficiency wages, workers who are caught shirking are not fired but instead they have to pay a~fine being fraction $f$ of current wage.

\begin{wylicz}
\item Write the Bellman equations for the values of $V_E$, $V_S$ and $V_U$ and find the 'No Shirking Condition'.
\item Draw the 'No Shirking Condition' line on a~graph. How does increase of $f$ influence unemployment? Explain the relation between monitoring of workers' effort and the value of the fine.
\item Find the value of $f$ for which the unemployment in this modified version of the model is lower than in the original Shapiro-Stiglitz model.
\end{wylicz}


%-------------------------------------------------------------------------------
\tytul{Problem}{}

\noindent%
Assume that the representative firm's profits are given by $\pi=(eL)^\alpha/\alpha-wL$ with $0<\alpha<1$, where $e$ is the effort level. The labour union objective function is $U=w-x$, where $x$ is the workers' outside option. Assume that a~firm and union bargain over wage, then firm chooses employment $L$ for this negotiated wage (as in the 'right-to-manage' model). Assume that $e=1$, so we do not consider efficiency wages.

\begin{wylicz}
 \item What is the level of $L$ chosen by firm for given wage $w$? What is the firm's profit?
 \item Assume that the bargaining power of the union is $\gamma$, where $0<\gamma<\alpha$. What is the negotiated wage?
 \item What is the value of $\partial (\ln w)/\partial\gamma$ for $\gamma=0$?
\end{wylicz}

Assume now that $e=[(w-x)/x]^\beta$, where $0<\beta<1$.

\begin{wylicz}
 \item[(d)] What is now the level of $L$ chosen by firm for given wage $w$? What is the firm's profit?
 \item[(e)] What is the negotiated wage in this setting? (check if your solution for $\beta=0$ is the same as in point (b)).
 \item[(f)] What is the value of $\partial (\ln w)/\partial\gamma$ for $\gamma=0$? Is this elasticity higher with efficiency wages or not?
\end{wylicz}

%-------------------------------------------------------------------------------
\tytul{Problem}{}

\noindent%
Assume that the firm's production function is $F(L)=L^\alpha/\alpha$, where $0<\alpha<1$ and the union objective function is $V=(L/N)\ln(w)+(1-L/N)\ln(B)$, where $B$ is the unemployment benefit.

\begin{wylicz}
\item Derive the firm's labour demand and iso-profit lines. Show them on appropriate graph.
\item Find the union indifference curves and show them on the same graph.
\item Assuming $\beta$ as the union bargaining power and the 'right-to-manage' model of wage setting, state the wage negotiation problem. Assume that the firm's fallback is zero.
\item Find the negotiated wage (may be for convenience expressed in log terms).
\item How does the union bargaining power affect negotiated wage and employment?
\item How does $\alpha$ influence the elasticity of labour demand with respect to wage? How does it influence wage and employment in equilibrium?
\item Show that the equilibrium is not Pareto efficient.
\end{wylicz}

%-------------------------------------------------------------------------------
\tytul{Problem}{}

\noindent%
Consider the search model of unemployment presented in the lecture. Using the graphs describing equilibrium, explain how each of the following events affect changes in wages, vacancies ratio and unemployment rate.

\begin{wylicz}
 \item A decrease of the separation rate
 \item An increase in the aggregate productivity
 \item An increase in the real interest rate
 \item An improvement in the matching efficiency
\end{wylicz}


%-------------------------------------------------------------------------------
\tytul{Problem}{}

\noindent%
Consider the search model of unemployment presented in the lecture. Suppose that the flow cost of a~vacancy $c$ and the imputed value of free time $z$ are functions of the wage rate $w$ (instead being exogenous). In particular, assume that $c=c_0w$ and $z=z_0w$.

\begin{wylicz}
 \item Determine the formula for job creation and wage setting.
 \item How do $\theta$ and wages in steady-state equilibrium react to productivity changes?
 \item Does a~continuous growth of productivity lead to a~decrease in the long run unemployment rate?
\end{wylicz}

%-------------------------------------------------------------------------------
\tytul{Problem}{}

\noindent%
Consider the search model of unemployment presented in the lecture. In contrast to what was assumed that the wage is bargained over by the worker and the firm, assume that it is set according to Shapiro-Stiglitz model of efficiency wages (including unemployment benefits $z$, like in Problem 5, Problem set \#2). Adopt notation from the search model and denote $p(\theta)$ -- probability of finding new job per unit of time, $s$ -- exogenous job destruction rate, $r$ -- discount rate and $\mu$ -- probability of detecting shirkers per unit of time.

\begin{wylicz}
 \item Write the new formula for wage setting.
 \item How does it differ from the original search model?
 \item Assume there is a~positive productivity shock. How does it affect labour market tightness ($\theta$), wages and unemployment in equilibrium? Are these effects different than in the original search model?
 \item Assume there is an increase of job destruction rate $s$. What happens to labour market tightness ($\theta$), wages and unemployment in equilibrium? Are these effects different than in the original search model?
\end{wylicz}

%-------------------------------------------------------------------------------
\tytul{Problem}{}

\noindent%
Consider the search model of unemployment presented in the lecture. Assume that the matching function is given by: $m(u,v)=\phi u^\alpha v^{1-\alpha}$, where $0<\alpha<1$. Let $s$ be the separation rate, $r$ -- real interest rate, $c$ -- flow cost of vacancy, $z$ -- unemployment benefits and $\beta$ -- workers' bargaining power.

\begin{wylicz}
\item Write the equation of unemployment rate dynamics. Find the steady-state unemployment rate as a~function of $\theta$. In the space $(u,\theta)$ draw the line $\dot u=0$ and a~vector field showing how the unemployment rate changes in given regions.
\item Write the Bellman equations for job creation of firms. Using the condition of wage setting, i.e. $w=(1-\beta)z+\beta(y+c\theta)$ find the equation of $\theta$ dynamics. In space $(u,\theta)$ draw the line $\dot \theta=0$ and a~vector field showing how $\theta$ changes in given regions.
\item Describe the stability of equilibrium in this model and show the effects of an increase of parameter $\phi$ (explain briefly the economic interpretation of this). Find the new steady-state equilibrium point and show the transitory dynamics.
\end{wylicz}


%-------------------------------------------------------------------------------
\tytul{Problem}{}

\noindent%
Assume that the government imposes a~minimum wage of $w_m$. We analyse the results of such policy using the search and matching model of unemployment presented in the lecture.

\begin{wylicz}
\item What is the upper boundary of the minimum wage that is consistent with this model (i.e. it still makes it profitable for the firms to enter the labour market).
\item How does the minimum wage influence the value of $\theta$ in equilibrium. Using the Beveridge curve (write appropriate formula) show how does it influence the unemployment in equilibrium.
\item Write the aggregate outflows from unemployment in the steady state as a~function of $\theta$. Show how the minimum wage influences the outflows from unemployment in the steady state.
\end{wylicz}


%-------------------------------------------------------------------------------
\tytul{Problem}{}

\noindent%
We analyse the search and matching model of unemployment presented in the lecture. The matching function is given by: $m(u,v)= m_0 u^\alpha v^{1-\alpha}$. Assume that the population grows at the constant rate of $n=\frac{\dot L}{L}$ and all new workers start their professional lives being unemployed.

\begin{wylicz}
\item Find the function $p(\theta)$.
\item Write the equation of unemployment rate dynamics accounting for population growth and find the equilibrium (steady-state) value of unemployment. How does it depend on $m_0$? Give economic interpretation for this.
\item Draw the Beveridge curve for this case in the $(u,v)$ plane. How does an increase in $n$ influence the equilibrium unemployment?
\item Suppose you observe a~one time increase of the population size (as opposed to constant growth over time) resulting from a~wave of immigration. What will be the short-run and long-run effects of this event on the rate of unemployment?
\end{wylicz}


%-------------------------------------------------------------------------------
\end{document}
